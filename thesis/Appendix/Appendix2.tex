\chapter{Quá trình xử lý dữ liệu BVH}
\label{appendix:BVHData}

\section{Cấu trúc khung xương của một nhân vật}
\label{appendix:BVHData:skeleton}

Một số tên khung xương nhân vật trong $75$ khung xương chuyển động bao gồm:

{
\small
\texttt{Hips},
\texttt{Spine},
\texttt{Neck},
\texttt{Head},
\texttt{RightShoulder},
\texttt{RightArm},
\texttt{RightForeArm},
\texttt{RightHand},
\texttt{LeftShoulder},
\texttt{LeftArm},
\texttt{LeftForeArm},
\texttt{LeftHand},
\texttt{RightUpLeg},
\texttt{RightLeg},
\texttt{RightFoot},
\texttt{RightToeBase},
\texttt{LeftUpLeg},
\texttt{LeftLeg},
\texttt{LeftFoot},
\texttt{LeftToeBase},
...
}

%\begin{itemize}
%	\item Left
%\end{itemize}

%\begincolcol}

\begin{figure}[H]
\centering
\includegraphics[height=10.5cm]{Bone}
\caption{Khung xương của một nhân vật}
\label{fig:Bone}
\end{figure}

\section{Cấu trúc tệp dữ liệu BVH}
\label{appendix:BVHData:BVHStructure}

File BVH (Biovision Hierarchy) là một định dạng file dữ liệu chứa thông tin về cấu trúc xương và dữ liệu về chuyển động của xương trong một hệ thống xương. File BVH bao gồm hai phần chính: phần khai báo cấu trúc xương và phần dữ liệu chuyển động của xương. 

\begin{itemize}
	\item \textbf{HIERARCHY}:
	
	\begin{itemize}
		\item Định nghĩa thành phần và tên các khung xương, vị trí khởi tạo của các khung xương ở vị trí T-pose (Tay diễn viên motion capture sẽ được duỗi thẳng thành hình chữ T).
		\item Định nghĩa quan hệ cha-con từ nút gốc đến các nút lá của một khung xương. Với nút gốc thường là cột sống ($\texttt{Spine}$).
		\item Định nghĩa các dữ liệu sẽ được ghi nhận như vị trí hoặc góc quay theo chiều $X, Y, Z$ của mỗi khung xương theo thời gian.
	\end{itemize}
	
	\item \textbf{MOTION}: Là chuỗi chuyển động theo từng khung hình, với mỗi khung hình là dữ liệu của từng xương về chuyển động theo thông tin về góc quay hoặc vị trí đã định nghĩa ở HIERARCHY.
\end{itemize}

\begin{figure}[htbp]
	\centering
	\begin{subfigure}{0.49\textwidth}
		\centering
		\includegraphics[width=\linewidth]{BVH1}
		\caption{HIERARCHY trong tệp BVH}
		\label{fig:BVH1}
	\end{subfigure}
	\hfill
	\begin{subfigure}{0.49\textwidth}
		\centering
		\includegraphics[width=\linewidth]{BVH2}
		\caption{MOTION trong tệp BVH}
		\label{fig:BVH2}
	\end{subfigure}
\end{figure}




%	\begin{figure}
	%		\centering
	%		\includegraphics[width=0.25\textwidth]{BVHFile}
	%	\end{figure}
	
$\mathbf{rotation}_i^{\operatorname{local}} = \{ \alpha ,\beta , \gamma \}$ lần lượt là góc quay quanh các trục $Z$ ,$Y$ , và $X$, góc quay tổng hợp ba trục trong không gian Euler là 

\begin{equation}
R = R_Z(\alpha) R_Y(\beta) R_X(\gamma)
\end{equation}

Trong đó:

1. \textbf{Ma trận quay quanh trục \(Z\)}:

\[
R_Z(\alpha) = 
\begin{bmatrix}
	\cos(\alpha) & -\sin(\alpha) & 0 \\
	\sin(\alpha) & \cos(\alpha) & 0 \\
	0 & 0 & 1
\end{bmatrix}
\]

2. \textbf{Ma trận quay quanh trục \(Y\)}:

\[
R_Y(\beta) = 
\begin{bmatrix}
	\cos(\beta) & 0 & \sin(\beta) \\
	0 & 1 & 0 \\
	-\sin(\beta) & 0 & \cos(\beta)
\end{bmatrix}
\]

3. \textbf{Ma trận quay quanh trục \(X\)}:

\[
R_X(\gamma) = 
\begin{bmatrix}
	1 & 0 & 0 \\
	0 & \cos(\gamma) & -\sin(\gamma) \\
	0 & \sin(\gamma) & \cos(\gamma)
\end{bmatrix}
\]

Để tính tọa độ chuyển động của một nhân vật, thực hiện phép toán sau:

\begin{equation}
	\mathbf{position}_{\text{global}} = R \cdot \mathbf{position}_{\text{local}} + \mathbf{t}
\end{equation}


\section{Quá trình chuyển góc quay Euler sang Quaternion}
\label{appendix:BVHData:QuaternionConvert}


Để tránh Gimbal lock, dữ liệu ở dạng góc quay Euler phải được chuyển sang dạng góc quay Quaternion. Trong đó góc quay từng Bone dạng Euler ZYX sang dạng Quaternion, mỗi Bone biểu diễn bằng 4 phần tử $q = (q_w, q_x, q_y, q_z)$, với các giá trị được tính như sau:
%	Mỗi Bone được biểu diễn thành: 

Đầu tiên, tính các giá trị $\cos$ và $\sin$ của một nửa góc quay cho mỗi trục:


\begin{itemize}
	\item $c_{\alpha} = \cos\left(\frac{\alpha}{2}\right), \quad s_{\alpha} = \sin\left(\frac{\alpha}{2}\right)$
	\item $c_{\beta} = \cos\left(\frac{\beta}{2}\right), \quad s_{\beta} = \sin\left(\frac{\beta}{2}\right)$
	\item $c_{\gamma} = \cos\left(\frac{\gamma}{2}\right), \quad s_{\gamma} = \sin\left(\frac{\gamma}{2}\right)$
\end{itemize}

Dựa trên các giá trị tính được ở trên, các thành phần của quaternion được tính như sau:


\begin{itemize}
	\item $q_w = c_{\alpha} c_{\beta} c_{\gamma} + s_{\alpha} s_{\beta} s_{\gamma}$
	\item $q_x = c_{\alpha} c_{\beta} s_{\gamma} - s_{\alpha} s_{\beta} c_{\gamma}$
	\item $q_y = c_{\alpha} s_{\beta} c_{\gamma} + s_{\alpha} c_{\beta} s_{\gamma}$
	\item $q_z = s_{\alpha} c_{\beta} c_{\gamma} - c_{\alpha} s_{\beta} s_{\gamma}$
\end{itemize}

Với quaternion $q$ đã tính, vị trí toàn cục của bone $\mathbf{p}_{\text{global}}$ được xác định bằng cách quay vị trí cục bộ $\mathbf{p}_{\text{local}}$ thông qua công thức:

\begin{equation}
	\mathbf{p}_{\text{global}} = q \cdot \mathbf{p}_{\text{local}} \cdot q^{-1} + \mathbf{t}
\end{equation}

với $\mathbf{t}$ là vị trí gốc của bone trong không gian toàn cục.