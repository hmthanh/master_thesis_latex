
\section{Kết luận}

\begin{frame}{Giải quyết các thách thức}
		\begin{enumerate}
			\item Quan hệ $n:n$ cử chỉ, âm thanh $\rightarrow$ Cross-Local Attention và Self-Attention
%			, chúng tôi giải quyết được quan hệ $1:n$ của cử chỉ và các điều kiện âm thanh, cảm xúc.
			\item Không cân xứng các đặc trưng $\rightarrow$ Mô hình Diffusion giúp mô hình có thể học được các đặc trưng có phân bố dữ liệu thật thấp.
%			\item Đảm bảo khả năng mở rộng (scalable) của hệ thống sinh cử chỉ để có thể ứng dụng trong thực tế.
				\item Thiếu đồng nhất dữ liệu $\rightarrow$ sử dụng Huber Loss
		\end{enumerate}
\end{frame}

\begin{frame}{Đóng góp chính}
	
	\begin{figure}
		\centering
		\includegraphics[width=0.9\linewidth]{TotalStage}
	\end{figure}
	
	\begin{itemize}
%		\item<1-> 
		\item \textbf{Sử dụng đặc trưng văn bản}: Transcribe âm thanh để có được văn bản (\textit{Công đoạn tiền xử lý}, \textit{xử lý đặc trưng} và \textit{kết hợp đặc trưng}), làm dữ liệu bổ sung trong Diffusion có điều kiện. 
		
%		\item<2-> 
		\item \textbf{Mở rộng mã nguồn}: \hyperlink{https://github.com/hmthanh/OHGesture}{github.com/OHGesture}, Mô hình pretrain: \hyperlink{https://huggingface.co/openhuman/openhuman}{Huggingface.co/openhuman}
		
%		\item<3->
		\item \textbf{Hệ thống trực quan hóa bằng Unity}: \hyperlink{https://github.com/DeepGesture/deepgesture-unity}{github.com/DeepGesture-Unity}.
		
%		\item<4-> 
		\item \textbf{Genea Leaderboard}: Paper xây dựng hệ thống chuẩn hóa
		
%		\item<5->
		\item \textbf{Đề xuất GestureScore} (Đánh giá cử chỉ bằng FGD): \hyperlink{https://github.com/GestureScore/GestureScore}{github.com/GestureScore}.
	\end{itemize}
\end{frame}

\begin{frame}[label=frame]{{Kết luận}}
	\begin{itemize}
		\item Mô hình \textbf{OHGesture}, có khả năng sinh cử chỉ chân thực, không chỉ trên các mẫu dữ liệu trong tập huấn luyện mà còn mở rộng được với những âm thanh không có trong dữ liệu huấn luyện.
		
		\item Sử dụng phương pháp \textbf{Classifier-free Guidance}, có thể điều khiển các điều kiện như cảm xúc, cử chỉ khởi tạo, có thể nội suy để suy luận ra giữa các cảm xúc khác nhau.
		
		\item Bài toán sinh cử chỉ còn rất mới, chúng tôi dự định sẽ dùng Fourier TT để trích xuất các đặc trưng về pha.
		
	\end{itemize}
	

%	In this slide, some important text will be
%	\alert{highlighted} because it's important.
%	Please, don't abuse it.
%	
%	\begin{block}{Remark}
%		Sample text
%	\end{block}
%	
%	\begin{alertblock}{Important theorem}
%		Sample text in red box
%	\end{alertblock}
%	
%	\begin{examples}
%		Sample text in green box. The title of the block is ``Examples".
%	\end{examples}
%	\hyperlink{appendix}{\beamerbutton{More on Appendix}}
\end{frame}